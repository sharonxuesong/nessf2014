% v0 draft with paragraph plans, first draft
% v1 improved upon feedbacks from Jason, and spell checks
% v2 my own proofread & improvements

\documentclass[12pt]{article}

%% Use pdfTex to insure the margin sizes in Windows

\usepackage{natbib}
%\usepackage{natbib,natbibspacing}
\setlength{\bibsep}{3pt} % spacing for bibliography items
\citestyle{aa}

\pdfpagewidth=8.5in
\pdfpageheight=11in

\setlength\topmargin{-0in}
\setlength\headheight{0in}
\setlength\headsep{0in}
\setlength\textheight{9in}
\setlength\textwidth{6.5in}
\setlength\oddsidemargin{0in}
\setlength\evensidemargin{0in}

\usepackage{mathrsfs}
\usepackage{amsmath,amssymb}
\usepackage{verbatim}
\usepackage{wrapfig}

\usepackage{enumitem}

% journal names
\usepackage{aas_macros}

\newcommand{\beq}{\begin{equation}}
\newcommand{\eeq}{\end{equation}}
\def\etal{~et~al.~}
%\def\mps{m~s$^{-1}$}
\def\mps{m/s}
\def\msini{M\sin{i}}
\def\mjup{M_{\rm Jup}}
\def\msol{M_{\odot}}
\def\degree{^{\circ}}
\def\leq{\leqslant}
\def\geq{\geqslant}
\def\kepler{{\it Kepler}}
\def\minerva{{\it Minerva}}
\def\hrs{HET/HRS}

\usepackage{graphicx}
%\usepackage[pdftex,bookmarks, % add hyperlinks
%colorlinks,
%plainpages=false]{hyperref}


\begin{document}

%\tableofcontents
%\newpage

%%%%%%%%%%%%%%%%%%%%%%%%%%%%%%%%%%%%%%%%%%%%%%%%%%%%%%%%%%%%%%%%%%%%%%%%%%%%%%%%%%%%%%%%%
\section{Objectives and Expected Significance}

% Finding low-mass exoplanets is important and interesting. Kepler has
% found a lot of candidates, but to measure the mass we need RV.




% But... current RV precision is limited by systematics and poorly
% understood stellar noises. To understand stellar noise, we must eliminate
% systematics as much as possible.



% We are going to address this BUT in this proposal: we have ALREADY
% identified several systematic contributions, and we propose to
% eliminate them.

\begin{itemize}[leftmargin=2.2em]
    \vspace{-3pt}
\item Improve the RV precision of Keck by eliminating the known
  systematics caused by telluric lines and applying better statistics.
    \vspace{-3pt}
\end{itemize}


\begin{itemize}[leftmargin=2.2em]
  \vspace{-3pt}
\item Prepare for the upgraded \hrs\ and the upcoming project \minerva.
  \vspace{-3pt}
\end{itemize}


% After this is done, there are plenty of wonderful science that will
% come out of this.
We can reanalyze the $\tau$ Ceti system \citep{tuomi2013}.





%%%%%%%%%%%%%%%%%%%%%%%%%%%%%%%%%%%%%%%%%%%%%%%%%%%%%%%%%%%%%%%%%%%%%%%%%%%%%%%%%%%%%%%%%
\vspace{-3pt}
\section{Methodology}


%---------------------------------------------------------------------------------------
\vspace{-3pt}
\subsection{Developing Data Reduction and Doppler Pipeline for the \\
  Upgraded HET/HRS and Minerva}\label{develop} 

%---------------------------------------------------------------------------------------
\vspace{-3pt}
\subsection{Improving the RV Precision of the Current HET/HRS}

\begin{comment}
\begin{wrapfigure}{r}{0.51\textwidth}
  \vspace{-35pt}
  \begin{center}
    \includegraphics[width=0.48\textwidth]{37605}
  \end{center}
  \vspace{-25pt}  
  \caption{Best Keplerian fit for the HD 37605 system (top panel:
    solid line) and the velocity residuals (bottom panel). We
    discovered HD 37605$c$ using \hrs\ data spanning $\sim$8 years
    (black dots) and also RVs from Keck and McDonald Observatory 2.1m
    telescope, but only \hrs\ data are shown here to highlight its
    precision and time baseline \citep{wang2012}. The bottom panel
    shows that the amplitude of the RV residuals decreased after the
    fine temperature control for the spectrograph room came online
    (epoch marked by dash-dotted line). The heights of the two grey
    regions are the RMS values before and after (9 and 6 m/s,
    respectively).}
  \vspace{-8pt}  
  \label{fit}
\end{wrapfigure}
\end{comment}


 
%%%%%%%%%%%%%%%%%%%%%%%%%%%%%%%%%%%%%%%%%%%%%%%%%%%%%%%%%%%%%%%%%%%%%%%%%%%%%%%%%%%%%%%%%
\vspace{-3pt}
\section{Relevance to NASA's Objectives and Support for NASA Missions}

Broadly, our investigation addresses one of the science
objectives of NASA SMD, ``Discover the origin, structure, evolution
and destiny of the universe and search for Earth-like planets".

More specifically, this proposal is directly and closely relevant to
the Astrophysics Research Program, theme (iii) Exoplanet Exploration,
in the solicitation:
\begin{itemize}[leftmargin=1.5em]
  \vspace{-3pt}
	\item ``to search for planets and planetary systems about
          nearby stars in our Galaxy": Our work with \hrs\ and
          \minerva\ is directly aimed at searching for exoplanets
          around nearby stars and \kepler\ targets (Section 2.1
          and 2.2).
          \vspace{-3pt}
	\item ``to determine the properties of those stars that harbor
          planetary systems": We will acquire high resolution spectra
          on planet host stars for, e.g. the \kepler\ stars, as
          required by the RV technique (Section 2.1). Our work on
          maintaining EOD also provides a database that enables
          statistical studies on stars that host exoplanets (Section
          2.3).
          \vspace{-3pt}
	\item ``to determine the percentage of planets that are in or
          near the Habitable Zone of a wide variety of stars and to
          measure their orbits": Project \minerva\ will find more
          Earths and super-Earths in or near the Habitable Zone of
          nearby stars. Future work with the upgraded \hrs\ will
          follow up potentially `rocky' \kepler\ planets in the
          Habitable Zone to reveal the complete architecture of their
          systems (Section 2.1 and 2.2).
          \vspace{-3pt}
\end{itemize}

Our work will also support current and future NASA missions and
enhance their scientific outcome: (1) {\bf the \textit{Kepler}
  mission}: The upgraded \hrs\ will directly support \kepler\ through
follow-up programs, including candidate validation/confirmation, TTV
follow-up, and outer planet discovery. {\bf (2) TESS:} The array of
four small telescopes of \minerva\ will be a great support for TESS by
following up its transiting planet candidates, since the TESS targets
are bright stars nearby. {\bf (3) JWST:} The Earths or super-Earths
with high transit probabilities discovered by \minerva\ are great
targets for JWST to characterize their planetary atmospheres.


%%%%%%%%%%%%%%%%%%%%%%%%%%%%%%%%%%%%%%%%%%%%%%%%%%%%%%%%%%%%%%%%%%%%%%%%%%%%%%%%%%%%%%%%%
\vspace{-3pt}
\section{Relation to PI's Other NASA Grants}

The PI of this proposal, Prof.~Jason Wright, has received NASA Keck
time to search for long-period planet and multiple planet systems.
The PI has also received NASA Keck time to follow up a
\kepler\ planetary system with a strong TTV signal (Co-I Eric
Ford). Both of these programs will of no doubt benefit from the
improved Keck RV precision, and the upgraded \hrs\ is expected to
follow up some if not all of the long-period planet systems and
\kepler\ TTV systems in the future.


%%%%%%%%%%%%%%%%%%%%%%%%%%%%%%%%%%%%%%%%%%%%%%%%%%%%%%%%%%%%%%%%%%%%%%%%%%%%%%%%%%%%%%%%%
%\newpage
%\addcontentsline{toc}{section}{References}
\vspace{-3pt}
\bibliographystyle{apj}	% (uses file "xxx.bst")
%{\footnotesize % smaller fonts for references
\bibliography{references} %}

\end{document}
