% NESSF 2014 proposal

\documentclass[12pt]{article}

%% Use pdfTex to insure the margin sizes in Windows

\usepackage{natbib}
%\usepackage{natbib,natbibspacing}
\setlength{\bibsep}{3pt} % spacing for bibliography items
\citestyle{aa} % apj type
%\citestyle{plain} ; number style

% try smaller section title fonts
\usepackage{sectsty}
\sectionfont{\large}
\subsectionfont{\normalsize}

\pdfpagewidth=8.5in
\pdfpageheight=11in

\setlength\topmargin{-0in}
\setlength\headheight{0in}
\setlength\headsep{0in}
\setlength\textheight{9in}
\setlength\textwidth{6.5in}
\setlength\oddsidemargin{0in}
\setlength\evensidemargin{0in}

\usepackage{mathrsfs}
\usepackage{amsmath,amssymb}
\usepackage{verbatim}
\usepackage{wrapfig}

\usepackage{enumitem}

% journal names
\usepackage{aas_macros}

\newcommand{\beq}{\begin{equation}}
\newcommand{\eeq}{\end{equation}}
\def\etal{~et~al.~}
%\def\mps{m~s$^{-1}$}
\def\mps{m/s}
\def\msini{M\sin{i}}
\def\mjup{M_{\rm Jup}}
\def\msol{M_{\odot}}
\def\mearth{M_{\oplus}}
\def\degree{^{\circ}}
\def\leq{\leqslant}
\def\geq{\geqslant}
\def\kepler{{\it Kepler}}
\def\minerva{{\it Minerva}}
\def\hrs{HET/HRS}
\def\keck{Keck/HIRES}

\usepackage{graphicx}
%\usepackage[pdftex,bookmarks, % add hyperlinks
%colorlinks,
%plainpages=false]{hyperref}

\begin{document}

%\tableofcontents
%\newpage

%%%%%%%%%%%%%%%%%%%%%%%%%%%%%%%%%%%%%%%%%%%%%%%%%%%%%%%%%%%%%%%%%%%%%%%%%%%%%%%%%%%%%%%%%
% title
\title{\vspace{-45pt} \bf \Large Finding the Lowest Mass Exoplanets with
  \\ Improved Radial Velocimetry \vspace{-6pt}}
\author{\normalsize Sharon Xuesong Wang}
\date{}
\maketitle

%%%%%%%%%%%%%%%%%%%%%%%%%%%%%%%%%%%%%%%%%%%%%%%%%%%%%%%%%%%%%%%%%%%%%%%%%%%%%%%%%%%%%%%%%
\vspace{-30pt}
\section{Overview}

% An overview of awesomeness of finding low-mass exoplanets.
The excellent synergy between NASA's \kepler\ mission and the
ground-based radial velocity (RV) surveys has made numerous
ground-breaking discoveries of exoplanets, including many interesting
low-mass and rocky planets (add ref: e.g., Howard, Pepe, Marcy). This
has brought the field of exoplanets into an exciting era with an
inceasing sample of small and potentially rocky planets (add ref:
Weiss) with the great promise towards the discovery of Earth analogs
in the near future.

% But... precise RV is yet to be more precise.
In the post-\kepler\ era, radial velocimetry will undoubtedly continue
to play a key role in validating \kepler\ candidates and measuring
their masses, as well as detecting exoplanets independently. However,
the current precision of radial velocimetry (0.5--1~\mps) acts as one
of the major limiting factors in detecting lower mass
exoplanets. Breaking this limit is critical for pushing the lower mass
boundary of the exoplanet ensemble. It is also an absolutely
necessary step towards finding Earth-mass ($\mearth$) exoplanets in
the habitable zone around Sun-like stars, which requires an RV
precision of $\sim$0.1 \mps.

% What we propose to do
\textbf{We propose to improve the RV precision of several leading RV
  instruments through correction of systematic errors, with the aim to
  find the lowest mass exoplanet.} Through our pilot study, we have
identified several contributing factors to RV systematic errors,
some of which are being recoganized and studied in detail \textit{for the
very first time}. These factors contribute to the RV error budget at
$\sim$1~\mps\ level, and thus they set the floor of long-term RV
precision at 1~\mps\ if not carefully studied and corrected for.

%%%%%%%%%%%%%%%%%%%%%%%%%%%%%%%%%%%%%%%%%%%%%%%%%%%%%%%%%%%%%%%%%%%%%%%%%%%%%%%%%%%%%%%%%
\vspace{-30pt}
\section{Expected Scientific Significance}

% Keck
The primary instrument we work with is the High Resolution Echelle
Spectrometer (HIRES) on Keck I. Among the RV discovered exoplanets,
\keck\ contributed to a majority of these discoveries. It also
contributed to a great number of mass measurements of confirmed
\kepler\ planets, especially \textit{most} of the low mass ones
\citep[e.g.,][]{gautier2012,gilliland2013,marcy2014}. However, its
current RV precision is limiting its ability to detect lower mass
planets or planets with the same mass but further out in orbit
\citep[e.g.,][]{marcy2014}. Our work will improve the RV precision of
\keck, and thus extend the lower mass limit of the current exoplanet sample.

% HET


%---------------------------------------------------------------------------------------
% RECYCLE BIN
\begin{comment}
We will work with the two leading RV instruments on 10-meter-class
telescopes: the High Resolution Echelle Spectrometer (HIRES) on Keck I
and the High Resolution Spectrograph (HRS) on the Hobby-Eberly
Telescope (HET).

Our work also has great synergy with two very high RV precision
instruments on smaller telescopes: CHIRON on the 1.5m SMARTS
telescope at CTIO and the upcoming project Minerva with an array of
four 0.7m telescopes.
  
The field of exoplanet is progressing in a fast pace towards the
discovery of Earth-like planets around other stars. During the past
decades, we have moved on from the age of booming discoveries of
Jupiter-mass exoplanets via radial velocimetry (add ref) to the
\kepler\ era where there are thousands of Earth- and super-Earth-size
exoplanet candidates (add ref). Moreover, great promises lie ahead with
future ground-based instruments (e.g., ESPRESSO; add ref) or space
missions (e.g., TESS; add ref).
\end{comment}
%---------------------------------------------------------------------------------------


%%%%%%%%%%%%%%%%%%%%%%%%%%%%%%%%%%%%%%%%%%%%%%%%%%%%%%%%%%%%%%%%%%%%%%%%%%%%%%%%%%%%%%%%%
\vspace{-3pt}
\section{The Identified in Precise Radial Velocimetry}
  
Importance of RV work, and the current limiting factors in precise RV:
`jitter' and what it means.

Provide examples of systems that suffer from this.

We have identified several aspects that can improve the precision of
radial velocimetry with two 10-meter-class telescopes, with great
synergy with smaller dedicated RV projects such as CHIRON and \minerva.

\begin{itemize}[leftmargin=2.2em]
    \vspace{-3pt}
\item Improve the RV precision of Keck through eliminating known
  systematics and improved statistics.
    \vspace{-3pt}
\end{itemize}



%%%%%%%%%%%%%%%%%%%%%%%%%%%%%%%%%%%%%%%%%%%%%%%%%%%%%%%%%%%%%%%%%%%%%%%%%%%%%%%%%%%%%%%%%
\vspace{-3pt}
\section{Proposed Work}

We propose to fix the following things:
\begin{itemize}
  \item Telluric lines.
  \item Wavelength-dependent statistical weighting.
  \item Iodine FTS validation.
  \item Data reduction and IP analysis.
\end{itemize}

Each `subsection' addresses one of the above.

We can reanalyze the $\tau$ Ceti system \citep{tuomi2013}. 

\begin{comment}
\begin{wrapfigure}{r}{0.51\textwidth}
  \vspace{-35pt}
  \begin{center}
    \includegraphics[width=0.48\textwidth]{37605}
  \end{center}
  \vspace{-25pt}  
  \caption{Plot 1.}
  \vspace{-8pt}  
  \label{fit}
\end{wrapfigure}
\end{comment}


 
%%%%%%%%%%%%%%%%%%%%%%%%%%%%%%%%%%%%%%%%%%%%%%%%%%%%%%%%%%%%%%%%%%%%%%%%%%%%%%%%%%%%%%%%%
\vspace{-3pt}
\section{Relevance to NASA's Objectives and Missions}

Broadly, our investigation addresses one of the science
objectives of NASA SMD, ``Discover the origin, structure, evolution
and destiny of the universe and search for Earth-like planets".

More specifically, this proposal is directly and closely relevant to
the Astrophysics Research Program, theme (iii) Exoplanet Exploration,
in the solicitation:
\begin{itemize}[leftmargin=1.5em]
  \vspace{-3pt}
\item ``to search for planets and planetary systems about
  nearby stars in our Galaxy": Our work with \hrs\ and
  \minerva\ is directly aimed at searching for exoplanets
  around nearby stars and \kepler\ targets (Section 2.1
  and 2.2).
  \vspace{-3pt}
\item ``to determine the properties of those stars that harbor
  planetary systems": We will acquire high resolution spectra
  on planet host stars for, e.g. the \kepler\ stars, as
  required by the RV technique (Section 2.1). Our work on
  maintaining EOD also provides a database that enables
  statistical studies on stars that host exoplanets (Section
  2.3).
  \vspace{-3pt}
\item ``to determine the percentage of planets that are in or near the
  Habitable Zone of a wide variety of stars and to measure their
  orbits": Project \minerva\ will find more Earths and super-Earths in
  or near the Habitable Zone of nearby stars. Future work with the
  upgraded \hrs\ will follow up potentially `rocky' \kepler\ planets
  in the Habitable Zone to reveal the complete architecture of their
  systems (Section 2.1 and 2.2).
  \vspace{-3pt}
\end{itemize}

Our work will also support current and future NASA missions and
enhance their scientific outcome: (1) {\bf the \textit{Kepler}
  mission}: The upgraded \hrs\ will directly support \kepler\ through
follow-up programs, including candidate validation/confirmation, TTV
follow-up, and outer planet discovery. {\bf (2) TESS:} The array of
four small telescopes of \minerva\ will be a great support for TESS by
following up its transiting planet candidates, since the TESS targets
are bright stars nearby. {\bf (3) JWST:} The Earths or super-Earths
with high transit probabilities discovered by \minerva\ are great
targets for JWST to characterize their planetary atmospheres.


%%%%%%%%%%%%%%%%%%%%%%%%%%%%%%%%%%%%%%%%%%%%%%%%%%%%%%%%%%%%%%%%%%%%%%%%%%%%%%%%%%%%%%%%%
\vspace{-3pt}
\section{Relation to PI's Other NASA Grants}

The PI of this proposal, Prof.~Jason Wright, has received NASA Keck
time to search for long-period planet and multiple planet systems.
The PI has also received NASA Keck time to follow up a
\kepler\ planetary system with a strong TTV signal (Co-I Eric
Ford). Both of these programs will of no doubt benefit from the
improved Keck RV precision, and the upgraded \hrs\ is expected to
follow up some if not all of the long-period planet systems and
\kepler\ TTV systems in the future.


%%%%%%%%%%%%%%%%%%%%%%%%%%%%%%%%%%%%%%%%%%%%%%%%%%%%%%%%%%%%%%%%%%%%%%%%%%%%%%%%%%%%%%%%%
%\newpage
%\addcontentsline{toc}{section}{References}
\vspace{-3pt}
\bibliographystyle{apj}	% (uses file "xxx.bst")
%{\footnotesize % smaller fonts for references
\bibliography{references} %}

\end{document}
