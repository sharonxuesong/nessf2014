% v0 draft with paragraph plans, first draft
% v1 improved upon feedbacks from Jason, and spell checks
% v2 my own proofread & improvements

\documentclass[12pt]{article}

%% Use pdfTex to insure the margin sizes in Windows

\usepackage{natbib}
%\usepackage{natbib,natbibspacing}
\setlength{\bibsep}{6pt}
\citestyle{aa}

\pdfpagewidth=8.5in
\pdfpageheight=11in

\setlength\topmargin{-0in}
\setlength\headheight{0in}
\setlength\headsep{0in}
\setlength\textheight{9in}
\setlength\textwidth{6.5in}
\setlength\oddsidemargin{0in}
\setlength\evensidemargin{0in}

\usepackage{mathrsfs}
\usepackage{amsmath,amssymb}
\usepackage{verbatim}
\usepackage{wrapfig}

\usepackage{enumitem}

% journal names
\usepackage{aas_macros}

\newcommand{\beq}{\begin{equation}}
\newcommand{\eeq}{\end{equation}}
\def\etal{~et~al.~}
%\def\mps{m~s$^{-1}$}
\def\mps{m/s}
\def\msini{M\sin{i}}
\def\mjup{M_{\rm Jup}}
\def\msol{M_{\odot}}
\def\degree{^{\circ}}
\def\leq{\leqslant}
\def\geq{\geqslant}
\def\kepler{{\it Kepler}}
\def\minerva{{\it Minerva}}
\def\hrs{HET/HRS}

\usepackage{graphicx}
%\usepackage[pdftex,bookmarks, % add hyperlinks
%colorlinks,
%plainpages=false]{hyperref}


\begin{document}

%\tableofcontents
%\newpage

%%%%%%%%%%%%%%%%%%%%%%%%%%%%%%%%%%%%%%%%%%%%%%%%%%%%%%%%%%%%%%%%%%%%%%%%%%%%%%%%%%%%%%%%%
\section{Objectives and Expected Significance}

The 9.2-meter Hobby-Eberly Telescope (HET) is poised to have comparable
power to Keck in its capabilities of exoplanet search. With multiple
upcoming upgrades for the telescope and its High-Resolution
Spectrograph (HRS), the throughput and radial velocity (RV) precision
of \hrs\ will be improved to the same level of Keck, which will enable
HRS to discover and follow up planets and \kepler\ candidates around
relatively faint stars that are far away.

Meanwhile, bright, nearby stars are very important targets for finding
Earth- or super-Earth-size planets in the Habitable Zone and/or with
large transit probability to enable potential studies on their
atmosphere. This is the primary science goal of project \minerva, an
array of four small telescopes that will perform dedicated RV
monitoring on a carefully-selected ensemble of nearby stars.

It is the broad theme of this proposal \textbf{to develop data
  reduction and Doppler pipelines for \hrs\ and \textit{Minerva} with
  the ultimate goal to find exoplanets around stars near and far}. We will
% 1. prepare for future
\begin{itemize}[leftmargin=2.2em]
    \vspace{-3pt}
\item Develop data reduction and Doppler (RV extraction) pipeline
  for the upgraded \hrs\ and \minerva.
    \vspace{-3pt}
\end{itemize}
Upon upgrade, the throughput of \hrs\ will be improved by a factor of
$\sim$5. HET will become the second telescope, besides Keck, that is
suitable for RV follow-up on the stars hosting planet candidates
discovered by \kepler. This will make the \kepler\ follow-up programs
more efficient. It will also benefit other planet search programs on
\hrs\ such as surveys on long-period planets and multiple-planet
systems.

As a dedicated RV survey project, \minerva\ is expected to discover
$\gtrsim 10$ Earth- to super-Earth-size planets with orbital period of
1--100 days around nearby stars, with 3--5 expected to be in the
Habitable Zones of their host stars
\citep{bottom2013,hogstrom2013}. The similarity between the spectral
data products of the upgraded \hrs\ and \minerva\ provides us the
opportunity to develop a single pipeline that will work equally well
for both.

% 2. improve current
\begin{itemize}[leftmargin=2.2em]
  \vspace{-3pt}
\item Improve the RV precision of the current \hrs.
  \vspace{-3pt}
\end{itemize}
By simply implementing the California Planet Survey (CPS) Doppler
pipeline without tailoring and refining it for the fiber-fed HRS, we have
already achieved $\sim$3 m/s \citep{johnson2011,wang2011,wang2012}. As
some essential parts of HRS will remain the same after the upgrade,
understanding the current bottleneck of its RV precision is essential
for ensuring a high-precision new \hrs. It will also make the
$\sim$10-year baseline \hrs\ archival data a very valuable asset for 
detecting multiple-planet systems and long-period planets, e.g. the
Solar system analogs and Jupiter analogs.

% 3. EOD
\begin{itemize}[leftmargin=2.2em]
    \vspace{-3pt}
\item Maintain and improve the Exoplanet Orbit Database (EOD; 
\citealt{wright2011})\footnote{This database (at exoplanets.org) is a
  compilation of all exoplanets that have quality peer-reviewed
  orbital measurements.} as a statistical study tool and an orbital
information database for exoplanet research.
  \vspace{-3pt}
\end{itemize}
We plan to incorporate the up-to-date \kepler\ catalog of planet
candidates into EOD. \textbf{This will put the \kepler\ planet candidates in
the context of all confirmed planets discovered by other programs and
methods.} EOD will also be the venue to keep close update on the
additional or refined orbital parameters of the \kepler\ planets that
are acquired through follow-up programs. This provides the community
with a complete and up-to-date collection of the reliable exoplanets
(and candidates) and their orbital parameters, and it is a powerful 
statistical tool for studies on planet occurrence rate and so on.


%---------------------------------------------------------------------------------------
% trial & error, recycle for all
\begin{comment}
This will make it the second telescope that is suitable for planet
discovery and characterization through RV observations on relatively
faint stars that are far away, such as the stars that host planet
candidates discovered by \kepler.
  
Among the 678 confirmed extrasolar planets as of today, 410 of them
were discovered through Radial Velocity (RV) measurement
(exoplanets.org; Wright\etal2011). The RV observation on stars near
and far plays an important role in both the discovery and the
characterization of exoplanets. In particular, RV follow-up plays an
important role in confirming the Kepler planet candidates, as well as
measuring their orbital and planetary parameters such as eccentricity
and mass.

Currently, the only $\sim$10-meter telescope that is capable of Kepler
follow-up in the Northern Hemisphere is the Keck telescopes, which has
an RV precision of $\sim$1 \mps. 

The launch of Kepler has dramatically increased the number of planet
candidates, and now there are 2740 potential planets in the Kepler
field, though so far only 105 planets are confirmed
(Batalha\etal2013). RV follow-up plays an important role in confirming
of the Kepler candidates, as well as measuring their orbital and
planetary parameters such as eccentricity and minimum mass, $\msini$.

% background
rv has been very successful. Kepler reshaped the field. rv follow-up
becomes important for kepler targets. need large telescope with high
rv precison. the only good enough telescope in the north is keck.

% how het is important
het is poised to be as good as keck. this will be the second 10m scale
telescope that has rv precision of 1m/s and below. our work aims to bring this
telescope to the game --- this is important for both Kepler follow-up and rv
exoplanet discovery on its own.

% section 3
key questions listed
under `Astrophysics' in the NASA's Science Mission Directorate
2007--2016 Science Plan, ``Create a census of extrasolar planets and
measuring their properties".

% section 2.2
The ability of fitting an pure iodine spectrum taken by the
spectrograph can reflect how well the IP is being modeled. The typical
$\chi_\nu^2$ value that we obtain for fitting iodine spectra with a
generic IP model, Gauss-Hermite polynomials, is about 2.5, while for
Keck/HIRES, the $\chi_\nu^2$ value is typically 1.05. The current
successful IP model for Keck/HIRES (sum of Gaussians) in the CPS
pipeline is the product of careful studies and numerous trials with IP
modeling. A better understanding of the IP of HRS will bring visible
improvements to its RV precision, since the IP modeling affects
directly several key procedures in the Doppler pipeline, such as the
creation of stellar spectrum template and the forward-modeling of the
observed stellar$+$iodine spectrum. We are at the beginning of this
endeavour, and we are confident that, with our experience with the
Keck/HIRES IP modeling, we can well characterize the HRS IP to achieve
a significant improvement in its RV precision.

% we are trying everything we can
We are also scrutinizing all aspects of the instrument and our
pipeline for places to improve. For example, we have recently
re-scanned the iodine cell used at HRS to assess the effects of cell
temperature change, which appears to be insignificant. The new iodine
scan, however, differs from the old one, and we are currently
investigating the reason behind this and the effects it brings.
\end{comment}
%---------------------------------------------------------------------------------------



%%%%%%%%%%%%%%%%%%%%%%%%%%%%%%%%%%%%%%%%%%%%%%%%%%%%%%%%%%%%%%%%%%%%%%%%%%%%%%%%%%%%%%%%%
\vspace{-3pt}
\section{Methodology}


%---------------------------------------------------------------------------------------
\vspace{-3pt}
\subsection{Developing Data Reduction and Doppler Pipeline for the \\
  Upgraded HET/HRS and Minerva}\label{develop} 

% background
The current \hrs, with its long term RV precision around 3--5 m/s
\citep{baluev2009,wang2011}, has made several discoveries of
exoplanetary systems (e.g.,
\citealt{wittenmyer2009,gettel2012,wang2012}). The key factor limiting
its science capabilities and preventing it from performing extensive
\kepler\ follow-up is its low throughput.

% explain the upgrade
However, the 2013 upgrade will greatly improve the throughput and will
enable \hrs\ to perform extensive \kepler\ follow-up. The telescope
will gain a factor of 1.4 in throughput as a result of better
tracking, better dome temperature control, and a better prime focus
optics. The throughput of HRS itself will also see an improvement of a
factor of 1.5--2 thanks to an added image slicer, a new cross
disperser, and a new optical configuration. Together, all the upgrades
on \hrs\ will improve its throughput in the wavelength range of the
iodine region by a factor of $\sim$5.

% what programs are we gonna pursue
We plan to have the data reduction and Doppler (RV extraction)
pipelines for the upgraded HRS ready when it is back online, which can
be late 2013 or early 2014. We will use the new \hrs\ to:
\begin{itemize}[leftmargin=1.5em]
      \vspace{-3pt}
  \item Follow up the \kepler\ Earth or super-Earth candidates to look
    for non-transiting gas giants further out in the system. The
    \kepler\ Earth or super-Earth candidates, especially the ones in
    Habitable Zones, are precious examples for answering questions
    such as the occurrence rate of the Solar system, the architecture
    of planetary systems with rocky planets, and the role that gas
    giant may have played in `fostering' the habitability of the inner
    rocky planets through dynamical evolution \citep{wetherill1994,horner2008}.
        \vspace{-3pt}
  \item Follow up the \kepler\ systems that show transit timing
    variations (TTVs), which are signposts of additional planets in the system
    (e.g. \citealt{steffen2013}). RV follow-up will reveal the full picture of
    the architecture of these systems, and is crucial for
    understanding occurrence rate of multiple-planet systems, the
    dynamics of planetary systems, as well as planet formation
    history.
        \vspace{-3pt}
  \item Use RV follow-up to validate/confirm the \kepler\ planet
    candidates, which will greatly increase the efficiency of RV
    validation/confirmation of \kepler\ candidates as HET joins Keck
    in this enormous and important task.
        \vspace{-3pt}
\end{itemize}

% what kind of data reduction pipeline pipeline is needed?
To be able to work with the upgraded \hrs, we need to make adjustments
and new developments on our current pipelines. The upgraded HRS will
produce spectral data in a different format. The newly added image
slicer will slice the fiber image into four parts, place them in
parallel, and feed them into the slit. Therefore, each of the spectral
orders will have four traces that are parallel and imaged close to
each other. This requires modifications and additions to the existing
spectral data reduction pipeline from bias correction and flat
fielding all the way to spectrum extraction. New elements will also be
needed for sky subtraction as a sky fiber is added, and
for combining four traces together to generate the final spectrum for
each order. Since we have set up the data reduction pipeline for the
current HRS (by adopting and modifying the REDUCE package;
\citealt{reduce2002}), we are confident about developing a
reliable new pipeline for the upgraded HRS.

% how about doppler pipeline, and upgrade's effect on RV precision
The upgrades of HRS will enhance its stability and
make it a better instrument for precise RV measurements in general. We
will modify our current Doppler pipeline so that it is
compatible with the new spectral format and the new instrumental
properties such as the instrumental profile (IP; also called the
point/line-spread function or the instrumental response function).
We will also be actively improving the Doppler pipeline to make sure
that it is of $\leq 1$ m/s precision once the upgraded \hrs\ is in
action (see the next section for more details). This work will be
greatly aided by our experience of implementing the CPS Doppler
pipeline for \hrs\ \citep{johnson2011,wang2012} and our close
collaboration with the CPS group.

% Introduction to Minerva. How \hrs\ pipeline also prepares for
% Minerva, and how Minerva differs. 
The data reduction and Doppler pipelines we develop for the upgraded
\hrs\ can also be applied to the spectral data produced by
\minerva\ (PI: John Johnson, Co-I: Jason Wright, Phil Murihead, Nate
McCrady). \minerva\ is a dedicated RV monitoring project on nearby
stars at Palomar Mountain to look for Earths and super-Earths that are
in the Habitable Zone and/or with high transit probability. It
consists of an array of four 0.7 meter telescopes with a vacuum-sealed
fiber-fed spectrograph, and it is designed to have $\leq 0.8$ m/s RV
precision \citep{bottom2013}. The spectral images will also have four
parallel traces for each spectral order, and our pipeline will be able
to combine these traces and measure their RV shift just like for \hrs,
or extract the four spectra and their RVs independently for each
telescope. Project \minerva\ has finished the preliminary design
review phase, and the first telescope is expected to be installed on
Palomar Mountain by the end of 2013.


%---------------------------------------------------------------------------------------
\vspace{-3pt}
\subsection{Improving the RV Precision of the Current HET/HRS}

% link between future and current, re-emphasize significance of
% improve current
To fully exploit the advantages brought by the upgrades of \hrs, it is
crucial to understand what is limiting its current RV precision, how
the upgrades would solve its current problems, and if more hardware
improvements are needed. The archival \hrs\ RV data with a
$\sim$10-year baseline provides a valuable and unique opportunity to
study the RV performances of fiber-fed spectrographs and their
long-term RV stability in general. A higher RV precision for the
current \hrs\ will also enable more science return on its archival
data, such as discovering long-period planets (Jupiter analogs) and
Solar system analogs.

% what we have achieved (185144 precision comp. to keck), and our
% experience with dealing with long-baseline hrs data
We have already implemented the CPS Doppler pipeline for \hrs. This is
the pipeline behind the successful CPS programs that discovered the
majority of RV exoplanets at Keck and Lick Observatories. Although this pipeline
was not designed to work with the fiber-fed HRS, it has achieved
long-term ($\sim$4 years) precision of 3.6 m/s on the RV standard star
$\sigma$ Draconis (HD 185144), where the long-term RV precision of
Keck on this star is 2.3 m/s \citep{wang2011}. Using this pipeline and data including
archival \hrs\ RVs as early as 2004, we discovered a second planet in
the HD 37605 system, HD 37605$c$, the 10th Jupiter analog in the peer
reviewed literature, with a 7.5-year period and $\msini = 3.4\ \mjup$
on a circular orbit \citep{wang2012}. Figure~\ref{fit} shows the
Keplerian orbital fit for the HD 37605 system and the \hrs\ RV data,
with residuals plotted in the lower panel.

\begin{wrapfigure}{r}{0.51\textwidth}
  \vspace{-35pt}
  \begin{center}
    \includegraphics[width=0.48\textwidth]{37605}
  \end{center}
  \vspace{-25pt}  
  \caption{Best Keplerian fit for the HD 37605 system (top panel:
    solid line) and the velocity residuals (bottom panel). We
    discovered HD 37605$c$ using \hrs\ data spanning $\sim$8 years
    (black dots) and also RVs from Keck and McDonald Observatory 2.1m
    telescope, but only \hrs\ data are shown here to highlight its
    precision and time baseline \citep{wang2012}. The bottom panel
    shows that the amplitude of the RV residuals decreased after the
    fine temperature control for the spectrograph room came online
    (epoch marked by dash-dotted line). The heights of the two grey
    regions are the RMS values before and after (9 and 6 m/s,
    respectively).}
  \vspace{-8pt}  
  \label{fit}
\end{wrapfigure}

% how 37605 reveals the problems and challenges (though temperature
% stablization helped).
Temperature stability in the spectrograph room of \hrs\ was identified
early on as one of the contributing factors to the RV systematic
errors, and this issue was resolved since the installation of a fine
temperature control system in March 2008 (J.~Bean, L.~Ramsey,
P.~McQueen private communications). We confirmed the improvement in RV 
precision as a result of this upgrade in our analysis with the HD
37605 data, which is illustrated in the lower panel of
Figure~\ref{fit}. The RMS of \hrs\ velocities with respect to the best
Keplerian fit of the HD 37605 system is 9 m/s for data before March
2008, and it is reduced to 6 m/s for data afterwards. Such improvement
is encouraging, and a closer look at the data and the intermediate
products of the Doppler pipeline reveals even more potential
contributors to the RV instability of \hrs.

% identified current problems and our plan to solve them:
% the ip problem = the core
One major issue is to model correctly the instrumental profile (IP) of
HRS. IP modeling is a crucial part of the precise RV work with iodine
calibration, as it affects directly several key procedures in the
Doppler pipeline, such as the creation of stellar spectrum template
and the forward-modeling of the observed stellar$+$iodine
spectrum. How well the IP is being modeled can be tested by fitting an
pure iodine spectrum taken by the spectrograph. The typical
$\chi_\nu^2$ value that we obtain for fitting iodine spectra with a
generic IP model (Gauss-Hermite polynomials) is about 2.5, while for
Keck/HIRES, the $\chi_\nu^2$ value is typically 1.05 (John Johnson
private communications). The current successful IP model for
Keck/HIRES (sum of Gaussians) in the CPS pipeline is the product of
careful studies and numerous trials with IP modeling.  A better
understanding of the IP of HRS will bring visible improvements to its
RV precision. We are at the beginning of this endeavour, and we are
confident that, with our experience with the Keck/HIRES IP modeling,
we can well characterize the HRS IP to achieve a significant
improvement in its RV precision.


%---------------------------------------------------------------------------------------
\vspace{-3pt}
\subsection{Maintaining and Improving the Exoplanet Orbit Database \label{database}}

% our purpose, and we are popular!
The Exoplanet Orbit Database (EOD; \citealt{wright2011}; exoplanets.org)
aims to provide a complete and reliable database of exoplanets orbital
information for both the scientific community and the general
public. We maintain a complete collections of {\it peer reviewed} orbital
parameters on all confirmed planets as well as the \kepler\ planet
candidates. Our website is the top 2 search result on Google for
keyword ``exoplanets" (after {\it Wikipedia} page), and our site
traffic is on average over 200 hits a day including 80 returned
viewers.

% Kepler!
Since most of the confirmed exoplanets today are discovered by
non-transiting methods, it is important to put the
\kepler\ planets/candidates into the context of all confirmed planets
discovered by various programs with different target-selection
strategies. In 2012, we included the first two releases of the
\kepler\ planet candidate catalogs in our database, and we plan to
incorporate very soon the newest releases \citep{batalha2013}. We will
update their validation/confirmation status and their newest orbital
information from follow-up programs such as RV, TTV, and imaging. We
will also keep their host star properties up-to-date as more
spectroscopic observations and analyses become available for
\kepler\ stars.

% pretty words on maintaining the catalog etc.
The compilation of all reliable planets/candidates with good orbital
parameters and host star characterization is crucial for estimating
occurrence rates of various types of planets hosted by a variety of
stars. With the EOD, we maintain and improve such a compilation. We
keep our database up-to-date diligently by examining the daily postings
on the arXiv, and extract new or refined orbital information. In the
era of \kepler\ when the number of planets/candidates are counted by
thousands, such statistical power will greatly advance our
understanding of the architecture and formation of exoplanet systems.

 
%%%%%%%%%%%%%%%%%%%%%%%%%%%%%%%%%%%%%%%%%%%%%%%%%%%%%%%%%%%%%%%%%%%%%%%%%%%%%%%%%%%%%%%%%
\vspace{-3pt}
\section{Relevance to NASA's Objectives and Support for NASA Missions}

Broadly, our investigation addresses one of the science
objectives of NASA SMD, ``Discover the origin, structure, evolution
and destiny of the universe and search for Earth-like planets".

More specifically, this proposal is directly and closely relevant to
the Astrophysics Research Program, theme (iii) Exoplanet Exploration,
in the solicitation:
\begin{itemize}[leftmargin=1.5em]
  \vspace{-3pt}
	\item ``to search for planets and planetary systems about
          nearby stars in our Galaxy": Our work with \hrs\ and
          \minerva\ is directly aimed at searching for exoplanets
          around nearby stars and \kepler\ targets (Section 2.1
          and 2.2).
          \vspace{-3pt}
	\item ``to determine the properties of those stars that harbor
          planetary systems": We will acquire high resolution spectra
          on planet host stars for, e.g. the \kepler\ stars, as
          required by the RV technique (Section 2.1). Our work on
          maintaining EOD also provides a database that enables
          statistical studies on stars that host exoplanets (Section
          2.3).
          \vspace{-3pt}
	\item ``to determine the percentage of planets that are in or
          near the Habitable Zone of a wide variety of stars and to
          measure their orbits": Project \minerva\ will find more
          Earths and super-Earths in or near the Habitable Zone of
          nearby stars. Future work with the upgraded \hrs\ will
          follow up potentially `rocky' \kepler\ planets in the
          Habitable Zone to reveal the complete architecture of their
          systems (Section 2.1 and 2.2).
          \vspace{-3pt}
\end{itemize}

Our work will also support current and future NASA missions and
enhance their scientific outcome: (1) {\bf the \textit{Kepler}
  mission}: The upgraded \hrs\ will directly support \kepler\ through
follow-up programs, including candidate validation/confirmation, TTV
follow-up, and outer planet discovery. {\bf (2) TESS:} The array of
four small telescopes of \minerva\ will be a great support for TESS by
following up its transiting planet candidates, since the TESS targets
are bright stars nearby. {\bf (3) JWST:} The Earths or super-Earths
with high transit probabilities discovered by \minerva\ are great
targets for JWST to characterize their planetary atmospheres.


%%%%%%%%%%%%%%%%%%%%%%%%%%%%%%%%%%%%%%%%%%%%%%%%%%%%%%%%%%%%%%%%%%%%%%%%%%%%%%%%%%%%%%%%%
\vspace{-3pt}
\section{Relation to PI's Other NASA Grants}

The PI of this proposal, Prof.~Jason Wright, has received NASA Keck
time to search for long-period planet and multiple planet systems. The
work proposed here will continue this effort as the upgraded \hrs\ is
expected to follow up some if not all of the systems in the future.
The PI has also received NASA Keck time to follow up a
\kepler\ planetary system with a strong TTV signal (Co-I Eric Ford). We
will continue such effort with the upgraded \hrs\ to follow up and
study more \kepler\ systems with TTVs (Section 2.1).


%%%%%%%%%%%%%%%%%%%%%%%%%%%%%%%%%%%%%%%%%%%%%%%%%%%%%%%%%%%%%%%%%%%%%%%%%%%%%%%%%%%%%%%%%
%\newpage
%\addcontentsline{toc}{section}{References}
\vspace{-3pt}
\bibliographystyle{apj}	% (uses file "xxx.bst")
\bibliography{references}

\end{document}
